
\section{Descriptions des besoins}

\subsection{Besoins Fonctionnels}
\subsubsection{Plateau de Jeu}

\begin{center}
    \centering
    \begin{tabular}[h]{|m{14cm}|m{2cm}|}
        \hline
        \rowcolor[HTML]{FFA8A8}
        \multicolumn{2}{|c|}{\textbf{Priorité 3/3}}                                                                                                                  \\
        \hline
        Besoins                                                                                                                                         & Avancement \\
        \hline
        • Les hexagones numérotés, chacun représentant une distance définie, par défaut : 16 kilomètres                                                 & \FAIT      \\
        • Définir les joueurs et leurs spécificités. Par exemple les nationalités possibles, le joueur qui joue le premier et celui qui a l’initiative. & \FAIT      \\
        • Pouvoir poser des unités sur la carte et à retenir celles qui ne sont plus présentes dans celle-ci                                            & \FAIT      \\
        • Pouvoir poser plusieurs unités sur un hexagone                                                                                                & \FAIT      \\
        • Définir une séquence de tour :
        \begin{itemize}
            \item Pouvoir alterner entre les joueurs
            \item Respecter l’ordre strict d’un tour
        \end{itemize}
                                                                                                                                                        & \FAIT      \\
        • Afficher un terminal qui permettra aux joueurs de donner des commandes d'attaque ou mouvement                                                 & \FAIT      \\
        • Être capable de faire des lancés de dés et d’appliquer des modificateurs. La majeure partie du jeu se base sur les dés                        & \FAIT      \\
        • Pouvoir déterminer quel joueur a l’initiative                                                                                                 & \NOP       \\
        \hline
    \end{tabular}
\end{center}

\begin{center}
    \centering
    \begin{tabular}[h]{|m{14cm}|m{2cm}|}
        \hline
        \rowcolor[HTML]{FFB72B}
        \multicolumn{2}{|c|}{\textbf{Priorité 2/3}}                                                                                                                                           \\
        \hline
        Besoins                                                                                                                                                                  & Avancement \\
        \hline
        • Différencier plusieurs types de terrain, les montages, les mers de sable et les crêtes par exemple                                                                     & \FAIT      \\
        • Créer une base pour les cartes d’évènements. Celles-ci étant très différentes, l’implémentation sera limitée aux évènements génériques (non spécifiques aux scénarios) & \NOP       \\
        • Pouvoir charger la carte du jeu à partir d’un fichier \textsc{txt} ou \textsc{json}                                         & \FAIT      \\
        \hline
    \end{tabular}
\end{center}

\begin{center}
    \centering
    \begin{tabular}[h]{|m{14cm}|m{2cm}|}
        \hline
        \rowcolor[HTML]{C0D8C0}
        \multicolumn{2}{|c|}{\textbf{Priorité 1/3}}                                                                                  \\
        \hline
        Besoins                                                                                                         & Avancement \\
        \hline
        • Ajouter les différents types d’indicateurs afin d’illustrer les villages, les villes et les oasis par exemple & \FAIT      \\
        \hline
    \end{tabular}
\end{center}

\subsubsection{API}

\begin{center}
    \centering
    \begin{tabular}[h]{|m{14cm}|m{2cm}|}
        \hline
        \rowcolor[HTML]{FFA8A8}
        \multicolumn{2}{|c|}{\textbf{Priorité 3/3}}                                                                                               \\
        \hline
        Besoins                                                                                                                      & Avancement \\
        \hline
        • Pouvoir échanger des informations basiques entre serveur et client                                                         & \FAIT      \\
        • Pouvoir convertir un état de la carte du jeu en format utilisable par le client pour pouvoir ensuite afficher le plateau   & \FAIT      \\
        • Envoyer un coup joué dans un format utile pour le serveur, pour pouvoir faire des éventuelles modifications sur le plateau & \FAIT      \\
        • Vérification des coups :
        \begin{itemize}
            \item Envoyer le coup joue au serveur
            \item Vérifier que le coup est valide
            \item Retourner une réponse positive ou négative. Si le coup est bon alors envoyer le nouvel état de la carte du jeu au client
        \end{itemize}
                                                                                                                                     & \FAIT      \\
        \hline
    \end{tabular}
\end{center}

\subsubsection{Unités}

\begin{center}
    \centering
    \begin{tabular}[h]{|m{14cm}|m{2cm}|}
        \hline
        \rowcolor[HTML]{FFA8A8}
        \multicolumn{2}{|c|}{\textbf{Priorité 3/3}}                                                                                                                            \\
        \hline
        Besoins                                                                                                                                                   & Avancement \\
        \hline
        • Définir l'unité comme interface, qui aura une morale, peut être perturbée (disrupted) et qui peut prendre des actions basiques comme attaquer et bouger & \FAIT      \\
        • Implémenter le système similaire à des points de vie (voir partie \textit{Depletion} des règles)                                                        & \FAIT      \\
        • Séparer les unités en catégories différentes : motorisées, à pied, mécanisées, cavalerie...                                                             & \FAIT      \\
        • Mettre en place les situations ou l'unité devient perturbée :                                                                                                        \\
        \hspace*{10mm} \- Résultat d'un combat                                                                                                                    & \FAIT      \\
        \hspace*{10mm} \- Trop de troupes sur un hexagone                                                                                                         & \FAIT      \\
        \hspace*{10mm} \- Fin d'un mouvement de nuit                                                                                                              & \NOP       \\
        \hspace*{10mm} \- Au début d'une phase de mouvement, si l'unité n'a pas accès à l'approvisionnement                                                       & \FAIT      \\
        \hline
    \end{tabular}
\end{center}

\begin{center}
    \centering
    \begin{tabular}[h]{|m{14cm}|m{2cm}|}
        \hline
        \rowcolor[HTML]{FFB72B}
        \multicolumn{2}{|c|}{\textbf{Priorité 2/3}}                                                \\
        \hline
        Besoins                                                                       & Avancement \\
        \hline
        • Permettre aux unités éligibles de s'entraîner et de faire des améliorations & \NOP       \\
        \hline
    \end{tabular}
\end{center}

% \subsubsection{Organisation de l'armée}


\subsubsection{Mouvement}

\begin{center}
    \centering
    \begin{tabular}[h]{|m{14cm}|m{2cm}|}
        \hline
        \rowcolor[HTML]{FFA8A8}
        \multicolumn{2}{|c|}{\textbf{Priorité 3/3}}                                                                                                       \\
        \hline
        Besoins                                                                                                                              & Avancement \\
        \hline
        • Déplacement case par case et enlever les points de mouvement correspondants de l'unité. Ceci pour limiter la capacité de mouvement & \FAIT      \\
        \hline
    \end{tabular}
\end{center}

\begin{center}
    \centering
    \begin{tabular}[h]{|m{14cm}|m{2cm}|}
        \hline
        \rowcolor[HTML]{C0D8C0}
        \multicolumn{2}{|c|}{\textbf{Priorité 1/3}}                                                                            \\
        \hline
        Besoins                                                                                                   & Avancement \\
        \hline
        • Le joueur Allié peut accélérer le mouvement de ses unités en utilisant le rail et le transport maritime & \NOP       \\
        • Avoir les types de mouvement speciaux: 
        \begin{itemize}
            \item Mouvement sur une route pour avancer plus vite
            \item Mouvement forcée pour bouger plus vite mais recevoir des effets négatifs à la fin du mouvement
        \end{itemize} & \NOP       \\
        \hline
    \end{tabular}
\end{center}

\begin{center}
    \centering
    \begin{tabular}[h]{|m{14cm}|m{2cm}|}
        \hline
        \rowcolor[HTML]{FFB72B}
        \multicolumn{2}{|c|}{\textbf{Priorité 2/3}}                                                                                                                                                                                                                                                                        \\
        \hline
        Besoins                                                                                                                                                                                                                                                                                               & Avancement \\
        \hline
        • Déplacement d'un point A à un point B, en parcourant le plus cours chemin. On utilisera l'algorithme {\tt Dijkstra} pour satisfaire la condition                                                                                                                                                    & \FAIT      \\
        • Appliquer un bonus de mouvement dépendant du terrain d'un hexagone. Il est plus facile de se déplacer sur une plaine plutôt que sur une montagne                                                                                                                                                    & \FAIT      \\
        • Permettre aux unités qui doivent se replier de bouger de 1-3 hexagones. Ce mouvement ne coûte pas de points de mouvement                                                                                                                                                                            & \FAIT      \\
        • Ajouter la mécanique du {\tt overrun} (fuite). Les unités capables de le faire peuvent, pendant leur phase de mouvement, attaquer des unités. Les défenseurs alors peuvent utiliser leurs phases de réaction pour bouger un certain nombre d'hexagones. La distance est définie par l'unité concernée & \NOP       \\
        \hline
    \end{tabular}
\end{center}

\begin{center}
    \centering
    \begin{tabular}[h]{|m{14cm}|m{2cm}|}
        \hline
        \rowcolor[HTML]{C0D8C0}
        \multicolumn{2}{|c|}{\textbf{Priorité 1/3}} \\
        \hline
        Besoins             & Avancement            \\
        \hline
        • Mouvement la nuit & \NOP                  \\
        \hline
    \end{tabular}
\end{center}

\subsubsection{Combat}

\begin{center}
    \centering
    \begin{tabular}[h]{|m{14cm}|m{2cm}|}
        \hline
        \rowcolor[HTML]{FFA8A8}
        \multicolumn{2}{|c|}{\textbf{Priorité 3/3}}                                                                                                                                                                                                                                                                                                                                        \\
        \hline
        Besoins                                                                                                                                                                                                                                                                                                                                                               & Avancement \\
        \hline
        • Pouvoir définir les unités participant à un combat                                                                                                                                                                                                                                                                                                                  & \FAIT      \\
        • Pouvoir déterminer la puissance de combat de l'armée composée de ces unités                                                                                                                                                                                                                                                                                         & \FAIT      \\
        • Définir les règles du combat, par exemple le fait que seules les unités/armées adjacentes peuvent entrer en combat, par la volonté de l'attaquant                                                                                                                                                                                                                   & \FAIT      \\
        • Pouvoir simuler le combat et donner les résultats :\\
        \hspace*{10mm} \- Déterminer les dégâts causés par une unité en divisant le nombre d'attaquants par le nombre de défenseurs de l'ennemie pour obtenir un ratio.                                                                                                                                                                                                       & \FAIT      \\
        \hspace*{10mm} \- Séparer les défenseurs en groupe de morale. Les unités avec le même morale se retrouvent dans le même groupe. Les résultats seront dans l'ordre descendant de morale. Par exemple si on a deux groupes de morale (de 1 et 2), alors les résultats du combat seront d'abord appliqués dans le groupe avec une morale de 2, puis a celui de morale 1. & \FAIT      \\
        \hspace*{10mm} \- Appliquer des éventuelles règles spéciales.                                                                                                                                                                                                                                                                                                         & \NOP       \\
        \hspace*{10mm} \- Si un hexagone contient que des unités de support, alors lancer un dé si l'attaquant le souhaite, pour tenter de capturer les unités.\newline Un résultat de 1-3 est un succès et un résultat de 4-6 veut dire que les unités de support sont détruites.                                                                                            & \FAIT      \\
        \hspace*{10mm} \- Amasser les dégâts et puis causer des dégâts aux unités adverses.\newline Les dégâts sont distribués parmi toutes les unités d'un groupe de morale.                                                                                                                                                                                                 & \FAIT      \\
        \hspace*{10mm} \- Enlever du plateau les unités détruites, et les ajouter dans la liste d'unités détruites du joueur concernée.                                                                                                                                                                                                                                       & \FAIT      \\
        \hspace*{10mm} \- Lancer un dé pour faire le test de morale des unités qui reste. Si le test échoue, alors l'unité est {\tt disrupted}.                                                                                                                                                                                                                                 & \FAIT      \\                        
        • Pouvoir simuler la retraite d'une armée si les spécifications le permettent, par exemple le terrain et la condition de l'armée est convenable, et si l'utilisateur le souhaite                                                                                                                                                                                      & \FAIT      \\
        \hline
    \end{tabular}
\end{center}

\begin{center}
    \centering
    \begin{tabular}[h]{|m{14cm}|m{2cm}|}
        \hline
        \rowcolor[HTML]{C0D8C0}
        \multicolumn{2}{|c|}{\textbf{Priorité 1/3}}                                                                         \\
        \hline
        Besoins                                                                                                & Avancement \\
        \hline
        • Si un {\tt Hex} contient plusieurs terrains, le défenseur doit pouvoir en choisir un pour sa défense & \NOP       \\
        \hline
    \end{tabular}
\end{center}

\subsubsection{Opérations Aériennes et Navales}

\begin{center}
    \centering
    \begin{tabular}[h]{|m{14cm}|m{2cm}|}
        \hline
        \rowcolor[HTML]{C0D8C0}
        \multicolumn{2}{|c|}{\textbf{Priorité 1/3}}                                                                                                                                       \\
        \hline
        Besoins                                                                                                                                                              & Avancement \\
        \hline
        • Pouvoir déterminer les différentes unités aériennes et navales ainsi que leurs spécificités                                                                        & \NOP      \\
        • Pouvoir déterminer les différentes cibles, par exemple des bases militaires ou les rivages(pour les opérations navales surtout), qu'ils peuvent cibler et attaquer & \NOP      \\
        • En ce qui concerne les opérations navales, ils peuvent effectuer des expéditions transportant des munitions ainsi que des unités/machines de guerre                & \NOP      \\
        \hline
    \end{tabular}
\end{center}

\subsubsection{Affichage}

\begin{center}
    \centering
    \begin{tabular}[h]{|m{14cm}|m{2cm}|}
        \hline
        \rowcolor[HTML]{FFA8A8}
        \multicolumn{2}{|c|}{\textbf{Priorité 3/3}}                                                                                                   \\
        \hline
        Besoins                                                                                                                          & Avancement \\
        \hline
        • Afficher le joueur dont c'est le tour                                                                                          & \FAIT      \\
        • Déterminer et afficher les informations de fin de partie et du vainqueur                                                       & \FAIT      \\
        • Afficher les différents marqueurs sur l'état de chaque composante du jeu, par exemple hors d'approvisionnement pour les unités & \FAIT      \\
        • Afficher le résultat et les informations à la fin du combat                                                                    & \FAIT      \\
        • Affiche un message d'erreur ou de refus quand une requête ou commande invalide est entrée.                                     & \FAIT      \\
        \hline
    \end{tabular}
\end{center}

\subsubsection{Réseaux}

\begin{center}
    \centering
    \begin{tabular}[h]{|m{14cm}|m{2cm}|}
        \hline
        \rowcolor[HTML]{FFA8A8}
        \multicolumn{2}{|c|}{\textbf{Priorité 3/3}}                                                                                                       \\
        \hline
        Besoins                                                                                                                              & Avancement \\
        \hline
        • Le jeu sera autour d'une architecture client-serveur qui puisse se déployer à travers Internet (pas seulement sur un réseau local) & \FAIT      \\
        \hline
    \end{tabular}
\end{center}

\subsection{Besoins Non-Fonctionnels}

\subsubsection{Affichage}
\begin{itemize}
    \item Le système doit être robuste aux erreurs de saisie et aux erreurs du serveur.
    \item Mettre en place une interface graphique affichant le plateau du jeu, les unités.
    \item Afficher un message d'attente au joueur qui ne joue pas.
\end{itemize}

\subsubsection{Système}
\begin{itemize}
    \item Le temps d'attente entre un coup proposé et sa validité évalué devront être de l'ordre de la seconde.
    \item Un chat sera créé pour communiquer avec l'adversaire.
\end{itemize}