\section{Conclusion}

Ce qu'il reste à faire :
\begin{itemize}
    \item Poser un cookie dans le client pour qu'il puisse reprendre une partie en cours s'il est déconnecté.
          Cela implique que la partie n'est pas directement détruite lorsqu'un joueur se déconnecte. Le serveur attendrait environs 3 minutes maximum avant de détruire la partie.
    \item Ajouter toutes les unités du jeu de base.
    \item Ajout de la mécanique {\tt overrun} qui permet l'attaque pendant la phase de mouvement.
    \item Ajout des phases : {\tt renforts}, {\tt allocation}, {\tt initiative}, {\tt aerial superiority}, {\tt supply attrition}, {\tt events}
          \begin{itemize}
              \item {\tt renforts} : Permet de ramener des unités détruites sur le terrain et les deux joueurs déterminent les unités disponibles comme renforts en cas d'attaque.
              \item {\tt allocation} : Permet d'activer les bases, construire des fortifications et entrainer les unités.
              \item {\tt initiative} : Lancer un dé permettant de déterminer quel joueur commence en premier.
              \item {\tt aerial superiority} : Permet de déployer ses unités aériennes puis de déterminer qui a l'avantage aérien.
                    La personne avec l'avantage peut faire des opérations aériennes.
              \item {\tt supply attrition} : Chaque joueur effectue un test de moral pour chaque unité qu'il possède afin de vérifier si son état devient {\tt disrupted} ou si elles deviennent isolées.
              \item {\tt events} : Permet de vérifier le status de tous les événements actifs. Chaque joueur déclare quels événements il veut provoquer.
          \end{itemize}
    \item Ajouter un temps de tour, c'est-à-dire que chaque joueur possède un certain temps pour jouer son tour. Actuellement si un jour ne fais rien, l'autre joueur doit attendre qu'il termine son tour.
    \item Ajouter tous les scénarios possibles du jeu de base.
\end{itemize}

L'un des problèmes majeurs rencontrés durant l'implémentation de ce projet est que les règles du jeu font environ 40 pages.
Cela qui nous a pris beaucoup de temps à comprendre et donc le début du développement du code a été retardé de quelques semaines.
De ce fait le projet n'est pas complétement fidèle aux règles du jeu de base.