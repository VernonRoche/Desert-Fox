\section{Conclusion}

Reste à faire:
\begin{enumerate}
    \item Poser un genre de cookie dans le client pour qu'il puisse reprendre sa partie en cours.
          Cela implique que la partie n'est pas directement détruite lorsqu'un joueur se déconnecte.
    \item Ajouter toutes les unités du jeu de base.
    \item Ajout de la mécanique {\tt overrun} qui permet l'attaque pendant la phase de mouvement.
    \item Ajout des phases: {\tt renforts}, {\tt allocation}, {\tt initiative}, {\tt aerial superiority}, {\tt supply attrition}, {\tt events}
          \begin{itemize}
              \item {\tt renforts}: Permet de ramener des unités détruites sur le terrain et les deux joueurs déterminent les unités disponibles comme renforts en cas d'attaque.
              \item {\tt allocation}: Permet d'activer les bases, construire des fortifications et entrainer les unités.
              \item {\tt initiative}: Lancer un dé permettant de déterminer quel joueur commence en premier.
              \item {\tt aerial superiority}: Permet de déployer ses unités aérienne puis de déterminer qui a l'avantage aérien.
                    La personne avec l'avantage peut faire des opérations aériennes.
              \item {\tt supply attrition}: Chaque joueur effectue un test de moral pour chaque unité qu'il possède afin de vérifier si son état devient {\tt disrupted} ou si elles deviennent isolées.
              \item {\tt events}: Permet de vérifier le status de tous les événements actifs. Chaque joueur déclare quels événements il veut provoquer.
          \end{itemize}
\end{enumerate}

Consigne : description de ce qu’il y aurait encore à faire, description des extensions possibles (et comment les réaliser).